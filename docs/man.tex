\documentclass[english]{article}

\usepackage[latin1]{inputenc}
\usepackage{latex2man}
\usepackage{babel}
\usepackage{verbatim}

\setDate{12/20/05}
\setVersion{0.2.3}

\begin{document}

% \begin{Name}{chapter}{name}{author}{info}{title}
\begin{Name}{1}{namedrop}{Jack Louis}{Network Tools}{namedrop command documentation}
%%%%%%%%%%

\Prog{namedrop} Version \Version\ is a (possibly, if compiled with c-ares) asynchronous DNS information enumeration
tool.

\end{Name}

\section{Synopsis}
%%%%%%%%%%

\Prog{namedrop}
   \oOpt{-4}
   \oOpt{-6}
\oOptArg{-a}{ number}
\oOptArg{-b}{ number-number}
\oOptArg{-c}{ characters}
   \oOpt{-d}
   \oOpt{-D}
   \oOpt{-e}
\oOptArg{-f}{ filename}
   \oOpt{-F}
   \oOpt{-h}
\oOptArg{-H}{ hostname}
   \oOpt{-r}
   \oOpt{-s}
   \oOpt{-S}
   \oOpt{-v}
   \oOpt{-V}
    \Arg{ target list}

\section{Description}
%%%%%%%%%%

\Prog{namedrop}:
	There are three basic modes of operation, dictionary brute force mode, general brute force mode and
reverse IP sweep mode. Dictionary brute force mode may use a (possibly compressed) file specified at the
command line, containing a hostnames separated by newlines.

\section{Options}
%%%%%%%%%%
\begin{Description}
\item[\Opt{-4}]
Imply that all questions are for IPV4, possibly requiring half the amount of questions
to be asked if IPV6 questions were also enabled.
\item[\Opt{-6}]
Does exactly the same thing as \Opt{-4} but it enables IPV6 not IPV4. This option can
also possibly serve to disambiguate the meaning of a hostname, that otherwise could return an IPV4
and a IPV6 record, in such a case the best guess is generally to use the IPV4 address record.
\item[\OptArg{-a}{ number}]
Concurrency for asynchronous resolver (how many questions to leave pending in parallel). This option
requires c-ares support to be enabled. Some experimentation is generally required to get this value at
a optimal setting, as it depends on the DNS cache(s) that are available.
\item[\OptArg{-b}{ number-number}]
Brute force mode, with mandatory argument in the form of \OptArg{-b}{ 3} or \OptArg{-b}{ 3-4} to brute force
hostnames from 1 (implied) to 3 characters, and from 3 to 4 characters, respectively.
\item[\OptArg{-c}{ characters}]
If inside of brute force mode, a list of characters that will be used to create hostnames from, such as
\OptArg{-c}{ abcdefghjiklmnopqrstuvwxyz-}. This option will also effect the random `wildcard' host
(if not specifically specified otherwise with \Opt{-H}) that is asked for to detect a default forward
record.
\item[\Opt{-d}]
Not yet implemented.
\item[\Opt{-D}]
Dictionary Brute force Mode (default).
\item[\Opt{-e}]
Display an exact output of data as returned by name server (only generally useful in a forward lookup mode).
If this option is selected CNAME records will be displayed so it is obvious what actual question generated
the A record output.
\item[\OptArg{-f}{ filename}]
Filename that will be used inside of a Dictionary mode brute force, the search order used to locate the file
is 1) locate a file in the current directory 2) append on .gz to the filename and locate it within the current
directory 2) prepend SHAREDIR (possibly /usr/local/share/namedrop or /usr/share/namedrop) onto the filename
3) append .gz to SHAREDIR/filename. Stated simply \OptArg{-f}{ foo} will become, ./foo, ./foo.gz,
/usr/share/namedrop/foo, and finally /usr/share/namedrop/foo.gz.
\item[\Opt{-F}]
Do not attempt to remove duplicate output, possibly display the same results over and over.
\item[\Opt{-h}]
Display command line program help.
\item[\OptArg{-H}{ hostname}]
Explicit hostname to be used during wildcard hostname detection.
\item[\Opt{-r}]
Reverse DNS sweeping mode (gather hostnames from a cidr address range), use \Opt{-4} or \Opt{-6} to specify
exactly what you want, otherwise the programs interpretation can be seemingly random.
\item[\Opt{-s}]
Output will be SQL, all data output will be in the form of SQL insert statements, and all non-data output
will be preceded with an SQL comment (--) so that a pipe may be used to interface to a command line SQL
monitor program like \Cmd{psql}{1}.
\item[\Opt{-S}]
Force synchronous DNS behavior, only useful if c-ares support is compiled in, as asynchronous questions
are default in that case.
\item[\Opt{-v}]
For each \Opt{-v} increase verbosity of program, useful when trying to locate a problem.
\item[\Opt{-V}]
Display program version.
\end{Description}

\section{Examples}
%%%%%%%%%%
\begin{Description}
\textbf{namedrop -ve4 domain.tld }
Dictionary Brute force the domain `domain.tld' for hostnames within asking for IPV4 records only.
If CNAMES are discovered \emph{-e} then display them with the actual A record returned. Also
display a small amount of information about things the program is doing.
\end{Description}

\section{Files}
%%%%%%%%%%
\begin{Description}
\item[\File{namedrop.conf}] The file containing the default configuration options for usage.
\end{Description}

\section{See Also}
%%%%%%%%%%
\Cmd{host}{1} \Cmd{dig}{1}

\section{Reporting Bugs}
%%%%%%%%%%
\begin{description}
Report Bugs to osace-users@lists.sourceforge.net
\end{description}

\section{Copyright}
%%%%%%%%%%
\begin{description}
\copyright\ 2005 Jack Louis \Email{jack@dyadsecurity.com}
This is free software; see the source for  copying  conditions.  There is NO warranty; not even for
MERCHANTABILITY or FITNESS FOR A PARTICULAR PURPOSE.
\end{description}

\LatexManEnd

\end{document}
